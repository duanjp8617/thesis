\chapter {Conclusion and Future Work}
\label{ch:conclusion}
\lhead{\chaptername \ \ref{ch:conclusion}.\ \emph{Conclusion and Future Work}} %

\section {Concluding Remarks}

This thesis discusses the design and implementation of \textit{ScalIMS}, \nfactor~and \netstar, which improve the scalability, performance of resilience functionalities and programability over existing NFV systems altogether.

\textit{ScalIMS} scales control-plane and data-plane service chains of IMS systems across geo-distributed datacenters by combining the benefits of proactive and reactive scaling. Evaluation of our prototype implementation on IBM SoftLayer cloud shows that \textit{ScalIMS} effectively reduces the total number of provisioned VNF instances while guaranteeing good traffic QoS.

\nfactor~leverages actor model to achieve transparent and highly efficient resilience functionalities. Our experiments show that \nfactor~achieves good scalability and high packet processing speed, as well as fast flow migration and failure recovery.

\netstar~represents first attempt to bring the future/promise abstraction to the NF dataplane for flow processing. Using \netstar, asynchronous programming in NFs is made easy and elegant, while good packet processing performance is still guaranteed. Our extensive evaluation shows that \netstar~can effectively simplify asynchronous programming asynchronous in NFs, and easily achieve line-rate packet processing for NFs.

\section {Future Work}

We identify two important research directions that are inspired by the study of this thesis.

\subsection{Bring Paxos-based Fault-tolerance to NFV}

An important application of both \nfactor~and \netstar~is to make NFs fault-tolerant. \nfactor~achieves fault-tolerance by constantly check-pointing the per-flow state of a flow actor to an replica actor, while \netstar~resists to failures by saving and updating important per-flow state to an external in-memory storage system.

Even though both methods can improve the failure-resilience performance of NFV system, they can only tolerate the failure of the NF instance that is correctly replicated. In practice, it is possible for both the replicated NF instance and the replica for storing important NF states to fail concurrently. In that case, the replicated NF will never be correctly recovered due to the loss of important NF states from the replica.

An important method for fighting against concurrent failures is to run Paxos algorithm \cite{lamport1998part}. With Paxos, a cluster of server instances can form a consensus group and the cluster can tolerate $N/2$ failured server instances for all $N$ server instances existed in the cluster. Existing work, like Crane \cite{cui2015p}, has discussed replicating general-purpose server system using Paxos algorithm.

However, replicating NF instances using Paxos algorithm raises a unique challenge over replicating general-purpose servers. NFs are usually design to work with a high input packet rate. A typical NF usually needs to process several million packets for each second. Such a high input rate poses a challenge for existing Paxos systems \cite{poke2015dare, 199299}, as they do not support such a high throughput.

Our work on \netstar~raises potential opportunities for bringing Paxos-based fault-tolerance to NFs. Implementing high-performance Paxos algorithm requires non-trivial asynchronous programming. With \netstar, we believe that the implementation of a Paxos replication system can be significantly simplified, while guaranteeing good performance. But to build a practical Paxos replication system for NFs, we still need to search for both architectural and algorithmic breakthroughs for Paxos algorithm.

\subsection{Building Formally Verified NFs}

An important research motivation for this thesis is to leverage modern programming paradigms to improve NFV system. In particular, this thesis discusses how to apply actor model and future/promise paradigm to improve the performance of resilience functionalities and the programability of high-performance NFs with asynchronous operations. Besides these two programming paradigms, another important paradigm which is worth further exploration is formal verification.

Building formally verified system software has draw important research attention in recent years \cite{199344, nelson2017hyperkernel, zaostrovnykh2017formally}. Formally verified software provides the strongest guarantee against various kinds of bugs and security risks. We believe that formal verification is extremely important for NFV technology as well. NFs are important access gateways of the underlying service, therefore any bugs and security risks of the NF may render the entire service protected by the NF useless.

A recent break-through in the formal verification of NFs is VigNAT \cite{zaostrovnykh2017formally}. VigNAT combines model-checking with theorem proving: it uses model-checking to check stateless part of VigNAT, and relies on theorem proving to verify the correctness of important data-structures used by VigNAT. By combining the result of model-checking and theorem proving, the VigNAT obtains a completely verified NF.

There are two major problems associated with VigNAT. First, VigorNAT treats the entire DPDK library \cite{dpdk} as the trusted computing base. VigorNAT assumes that DPDK is correct and does not care about potential bugs in DPDK library. However, DPDK itself is an extremely complicated library. It is used to retrieve network packets from the NIC card and serves as the the core of many modern, high-performance NFs. To further improve the reliability of formally verified NFs, the DPDK library must be removed from the trusted computing base. Second, the verification of the data-structures used by VigNAT only preserves the correctness up to the semantics of C programming language. There is no guarantee that the compiled machine code is correct.

We believe that to further improve the reliability of formally verified NFs, an effective approach is to adopt recent advancement of formal verification techniques, especially the Verified Software Toolchain (VST) project \cite{vst}. The VST project applies formal verification directly to C code, and guarantees the correctness of the compiled machine code with the help of CompCert compiler \cite{leroy2006formal}. Also, a recent project called ixy \cite{ixy} demonstrates the implementation of a simple user-space packet processing library, whose functionality is similar to DPDK. By following the design approach of ixy, it is possible to build a formally verified user-space packet processing library and remove DPDK library from the trusted computing base.

%Implementing high-performance Paxos algorithm inside NFs requires non-trivial asynchronous programming, which
